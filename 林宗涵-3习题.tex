\section{习题3}

\newpage
\subsection{3.1}
利用无穷小的变动,导出下列各平衡判据(假设总粒子数不变,且 $S>0$):
(i) 在 $U$ 及 $V$ 不变的情形下,平衡态的 $S$ 极大;
(ii) 在 $S$ 及 $V$ 不变的情形下,平衡态的 $U$ 极小;
(iii) 在 $S$ 及 $U$ 不变的情形下,平衡态的 $V$ 极小;
(iv) 在 $H$ 及 $p$ 不变的情形下,平衡态的 $S$ 极大;
(v) 在 $S$ 及 $p$ 不变的情形下,平衡态的 $H$ 极小;
(vi) 在 $T$ 及 $V$ 不变的情形下,平衡态的 $F$ 极小;
(vii) 在 $F$ 及 $T$ 不变的情形下,平衡态的 $V$ 极小;
(viii) 在 $T$ 及 $p$ 不变的情形下,平衡态的 $G$ 极小。

\newpage
\subsection{3.2}
用内能判据导出热、力学和相变平衡条件。

\newpage
\subsection{3.3}
从不等式(3.4.16)出发,选 $T$ 和 $p$ 为独立变量,导出稳定条件:
$$c_p > 0, \quad \frac{c_p}{T} \left( \frac{\partial v}{\partial p} \right)_T + \left( \frac{\partial v}{\partial T} \right)_p^2 < 0.$$

\newpage
\subsection{3.4}
证明:
(i) $\left( \frac{\partial \mu}{\partial T} \right)_{V,N} = - \left( \frac{\partial S}{\partial N} \right)_{T,V}$;
(ii) $\left( \frac{\partial \mu}{\partial p} \right)_{T,N} = \left( \frac{\partial V}{\partial N} \right)_{T,p}$;
(iii) $\left( \frac{\partial U}{\partial N} \right)_{T,V} - \mu = - T \left( \frac{\partial \mu}{\partial T} \right)_{V,N}$.

\newpage
\subsection{3.5}
令 $c_\beta^\alpha$ 为 $\alpha$ 相的两相平衡比热,其定义为:在保持 $\alpha$ 相与 $\beta$ 相两相平衡的情形下,1 mol(或 1 g)$\alpha$ 相物质温度升高 1 K 所吸收的热量。
(i) 根据上述定义,证明:
$$c_{\beta}^{\alpha} = c_{p}^{\alpha} - \frac{\lambda_{\alpha\beta}}{v^{\alpha} - v^{\beta}} \left( \frac{\partial v^{\alpha}}{\partial T} \right)_{p},$$
及
$$c_{\alpha}^{\beta} = c_{p}^{\beta} - \frac{\lambda_{\alpha\beta}}{v^{\alpha} - v^{\beta}} \left( \frac{\partial v^{\beta}}{\partial T} \right)_{p}.$$
(ii) 证明:
$$\frac{d\lambda_{\alpha\beta}}{dT} = \frac{\lambda_{\alpha\beta}}{T} + c_{\beta}^{\alpha} - c_{\alpha}^{\beta}.$$
(iii) 若 $\alpha$ 相是蒸气,并设可近似当作理想气体;$\beta$ 相是液相。证明上述 $c_{\beta}^{\alpha}$ 的公式可以简化为
$$c_{\beta}^{\alpha} = c_{p}^{\alpha} - \frac{\lambda_{\alpha\beta}}{T}.$$
由上式可以说明饱和蒸气的两相平衡比热 $c_{\beta}^{\alpha}$ 在什么条件下是负。

\newpage
\subsection{3.6}
利用下面的可逆循环过程:
(a) 在 $T, p$ 下由 $\alpha$ 相转变为 $\beta$ 相;
(b) 在保持 $\beta$ 相与 $\alpha$ 相平衡的情形下,由 $T, p$ 变为 $T + dT, p + dp$;
(c) 在 $T + dT, p + dp$ 下由 $\beta$ 相转变为 $\alpha$ 相;
(d) 在保持 $\alpha$ 相与 $\beta$ 相平衡的情形下,由 $T + dT, p + dp$ 回到 $T, p$ 态。

计算每一步内能的改变和熵的改变,使整个循环过程的改变为零,即
$$\sum \Delta U = 0 \text{ 和 } \sum \Delta S = 0,$$
导出 $dp/dT$ 及 $d\lambda/dT$ 的公式:
$$\frac{dp}{dT} = \frac{\lambda_{\alpha\beta}}{T(v^{\beta} - v^{\alpha})},$$
$$\frac{d\lambda_{\alpha\beta}}{dT} = \frac{\lambda_{\alpha\beta}}{T} + C_p^{\beta} - C_p^{\alpha}.$$
(注:原文件此处为 $C_p^a - C_a^a$,根据上下文和公式(7),应为 $C_p^{\beta} - C_p^{\alpha}$)

\newpage
\subsection{3.7}
两相平衡共存系统的 $C_p, \alpha$ 和 $\kappa_T$ 都是无穷大,试说明之。

\newpage
\subsection{3.8}
证明范德瓦耳斯气体在 $T<T_c$ 的 $p-v$ 等温线上 的极小点 $M$ 与极大点 $N$ 的轨迹为
$$pv^3 = a(v-2b).$$

\newpage
\subsection{3.9}
从范德瓦耳斯对比物态方程(3.10.14)出发,在临界点的邻域引入无量纲变量
$$\pi=\tilde{p}-1=(p-p_c)/p_c, \quad \phi=\tilde{v}-1=(v-v_c)/v_c, \quad \tau=\tilde{T}-1=(T-T_c)/T_c,$$
利用在临界点的邻域 $\pi, \phi$ 和 $\tau$ 都是小量,试由 (3.10.14) 导出 (3.10.18) 式,即由 $\left( \tilde{p} + \frac{3}{\tilde{v}^2} \right)(3\tilde{v}-1)=8\tilde{T}$ 导出
$$\pi = 4\tau - 6\tau\phi - \frac{3}{2}\phi^3.$$
(注:原文件物态方程中为 $\frac{3}{\tilde{v}}$,根据范德瓦耳斯对比方程标准形式,应为 $\frac{3}{\tilde{v}^2}$)

\newpage
\subsection{3.10}
由 (3.10.18) 式出发,证明范德瓦耳斯气体的临界指数 $\beta = \frac{1}{2}, \delta = 3, \gamma = 1$.