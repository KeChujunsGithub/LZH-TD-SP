\section{相变的热力学理论}
相变是自然界中广泛存在的一类现象.
本章介绍相变的热力学理论.
首先,根据热力学第二定律判断不可逆过程方向的结论,推导出判断热力学系统平衡态的普遍准则,即热动平衡判据.
然后根据平衡判据,导出具体的平衡条件与稳定条件.以上这些是基本定律的推论与发展.
接着将介绍相图;
讨论气-液相变、正常-超导相变、相变的分类.
然后简单介绍朗道的二级相变理论.
最后对临界现象的相关概念及平均场近似理论作一简介.

%%%%%%%%%%%%%%%%%%%%%%%%%%%%%%%%%%%%%%%%%%%%%%%%%%%%%%%%%%%%%%
\subsection{热动平衡判据}
热动平衡判据是判断热力学系统是否处于平衡态的普遍准则,它是热力学第二定律关于判断不可逆过程方向的普遍准则的推论.
%%%%%%%%%%%%%%%%%%%%%%%%%%%%%%%
\subsubsection{}
热动平衡理论不是这样的好学的号码








%%%%%%%%%%%%%%%%%%%%%%%%%%%%%%%
\subsubsection{}







%%%%%%%%%%%%%%%%%%%%%%%%%%%%%%%
\subsubsection{几点说明}

以上所得到的几种热动平衡判据,都是热力学第二定律关于不可逆过程进行方向的结论的推论.




%%%%%%%%%%%%%%%%%%%%%%%%%%%%%%%%%%%%%%%%%%%%%%%%%%%%%%%%%%%%%%
\subsection{粒子数可变系统}


%%%%%%%%%%%%%%%%%%%%%%%%%%%%%%%%%%%%%%%%%%%%%%%%%%%%%%%%%%%%%%
\subsection{热动平衡条件}
热动平衡判据:判断热力学系统是否达到平衡态的普遍准则
热动平衡条件:维持热力学系统平衡态的具体条件

3.1 中根据热力学第二定律所导出的热动平衡判据是判断系统是否认到平衡态的普遍准则,数学上是以某个热力学函数取条件极值的形式来表达的.
本节将从热动平衡判据出发导出热动平衡条件(或简称平衡条件),它是指维持热力学系统平衡态的具体条件,有如下四种:
(1)热平衡条件:物体内部各部分之间不发生热量交换的条件.
(2)力学平衡条件:物体内部各部分之间不发生宏观位移的条件.
(3)相变平衡条件:各相之间不发生物质转移(即不发生相变)的条件.
(4)化学平衡条件:化学反应不再进行的条件.
上述各平衡条件是针对不同类型的具体过程达到平衡的条件,但平衡判据是普遍的,对任何过程均适用.另外,这些平衡都是宏观意义上的,微观上分子仍在不断地运动,并非绝对的静止,这也是为什么称为热动平衡的原因.

\subsubsection{用熵判据推导平衡条件}
热平衡条件
两个子系统的温度相等.
如果$T_1\neq T_2$,表示两相没有达到热平衡,必然会发生变化.
在孤立系的条件下,变化应向着熵增加的方向进行.
为简单,可以令$\delta V_1=0$,$\delta N_1=0$
由,变化应使
\begin{equation}
    \mathrm{d}S=\left( \frac{1}{T_1}-\frac{1}{T_2} \right) \mathrm{d}U_1>0
\end{equation}


力学平衡条件
两个子系统达到力学平衡时,其压强相等
如果 $p_1\neq p_2$,表示两相没有达到力学平衡,必然会发生变化,
在孤立系的条件下,变化应向着熵增加的方向进行.
假设满足热平衡条件$T_1=T_2=T$,且$\delta N_1=0$
由,变化应使
\begin{equation}
    \mathrm{d}S=\frac{1}{T}\left( p_1-p_2 \right) \mathrm{d}V_1>0
\end{equation}
若 $p_1>p_2$,则应有 d$V_1>0$,表示压强高的子系统要膨胀,压强低的子系统要缩小,即$dV_2=-dV_1$,


相变平衡条件
一个化学纯(即单元系)的物体系的两相达到平衡时,两相的化学势相等.
假设相变平衡条件不满足,即$\mu_1\neq u_2$,这时系统必会发生变化。
为了集中考查两相之间物质的转移。
在孤立系的条件下,变化应向着熵增加的方向进行.

由,变化应使
\begin{equation}
    \mathrm{d}S=-\frac{1}{T}\left( \mu _1-\mu _2 \right) \mathrm{d}N_1>0
\end{equation}
若 $\mu_1>\mu_2$,则应有 d$N_1<0$ 表示化学势高的子相的物质将减少, 而化学势低的子相物质将增加,即$\mathrm{d}N_2=-\mathrm{d}N_1$。也就是说,相变过程是物质从化学势高的相向化学势低的相转变.


