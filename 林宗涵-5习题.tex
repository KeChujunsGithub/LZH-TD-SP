\section{习题}

\newpage
\subsection{5.1}
证明(5.3.15)与(N1)、(N2)、(N3)诸式(见原书239页注)相等。

\newpage
\subsection{5.2}
若选电流密度 $J$ 与热流密度 $J_q$ 作为热力学流,其共轭热力学力 $X_j$ 与 $X_q$ 由原书(5.3.16a)与(5.3.16b)给出,相应的唯象方程为(5.3.18a)、(5.3.18b)。试以此为基础,导出温差电效应的汤姆孙第一关系与第二关系。(参看王竹溪著,《热力学》(第二版),北京大学出版社,2005年,413页)

\newpage
\subsection{5.3}
如果像§5.3一样选电流密度 $J$ 与熵流密度 $J_s$ 作为热力学流,其共轭热力学力 $X_1$ 与 $X_2$ 由 (5.3.22a) 与 (5.3.22b) 给出,相应的唯象方程为 (5.3.23a)、(5.3.23b)。试以此为基础,从熵平衡方程导出汤姆孙第一关系与第二关系(不同于§5.3,那里是从能量平衡方程的分析导出两个关系)。

(参看S.R.德格鲁脱,P.梅休尔,《非平衡态热力学》,陆全康译,上海科技出版社,1981年,第298-303页。)

\newpage
\subsection{5.4}
由 (5.3.15),证明熵产生率 $\theta \geq 0$。
提示:从与 (5.3.15) 等价的公式 (N1) 证明比较简单。