\section{习题1}



\newpage
\subsection{1.1}
设三个函数 $f, g, h$ 都是二维立变量 $x, y$ 的函数,证明:
(i) $$ \left( \frac{\partial f}{\partial g} \right)_h = 1 / \left( \frac{\partial g}{\partial f} \right)_h; $$
(ii) $$ \left( \frac{\partial f}{\partial g} \right)_x = \frac{\partial f}{\partial y} / \frac{\partial g}{\partial y}; $$
(iii) $$ \left( \frac{\partial y}{\partial x} \right)_f = - \frac{\partial f}{\partial x} / \frac{\partial f}{\partial y}; $$
(iv) $$ \left( \frac{\partial f}{\partial g} \right)_h \left( \frac{\partial g}{\partial h} \right)_f \left( \frac{\partial h}{\partial f} \right)_g = -1; $$
(v) $$ \left( \frac{\partial f}{\partial x} \right)_g = \frac{\partial f}{\partial x} + \frac{\partial f}{\partial y} \left( \frac{\partial y}{\partial x} \right)_g. $$

注:$$ \frac{\partial f}{\partial x} $$ 指 $$ \left( \frac{\partial f}{\partial x} \right)_y, \frac{\partial f}{\partial y} $$ 指 $$ \left( \frac{\partial f}{\partial y} \right)_x $$。凡不指明求偏微商时的不变量的,均指原设函数关系下的偏微商。

\newpage
\subsection{1.2}
设四个函数 $f, g, h, k$ 都是二独立变量 $x, y$ 的函数,并以符号 $\frac{\partial (f, g)}{\partial (x, y)}$ 代表其雅可比行列式:

$$\frac{\partial (f, g)}{\partial (x, y)} = \left| \begin{array}{cc}
\frac{\partial f}{\partial x} & \frac{\partial f}{\partial y} \\
\frac{\partial g}{\partial x} & \frac{\partial g}{\partial y}
\end{array} \right| = \frac{\partial f}{\partial x} \frac{\partial g}{\partial y} - \frac{\partial f}{\partial y} \frac{\partial g}{\partial x}$$

证明:
(i) $\frac{\partial (f, g)}{\partial (h, k)} = \frac{\partial (f, g)}{\partial (x, y)} / \frac{\partial (h, k)}{\partial (x, y)}$;
(ii) $\frac{\partial (f, g)}{\partial (x, y)} = 1 / \frac{\partial (x, y)}{\partial (f, g)}$;
(iii) $\left( \frac{\partial f}{\partial g} \right)_h = \frac{\partial (f, h)}{\partial (g, h)}$;
(iv) $\left( \frac{\partial f}{\partial g} \right)_h = \frac{\partial (f, h)}{\partial (x, y)} / \frac{\partial (g, h)}{\partial (x, y)}$;
(v) $\left( \frac{\partial f}{\partial x} \right)_g = \frac{\partial (f, g)}{\partial (x, y)} / \frac{\partial g}{\partial y}$

\newpage
\subsection{1.3}
证明理想气体的膨胀系数 $\alpha$、压强系数 $\beta$ 及等温压缩系数 $\kappa_T$ 分别为 $\alpha = \beta = 1/T$, $\kappa_T = 1/p$。

\newpage
\subsection{1.4}
证明任何一个有两个独立变量 $T, p$ 的 $p-V-T$ 系统,其物态方程可由实验测得的膨胀系数 $\alpha$ 及等温压缩系数 $\kappa_T$ 根据下列积分求得:

$$\ln V = \int (\alpha dT - \kappa_T dp).$$

再应用这个公式和题1.3的结果,求理想气体的物态方程。

\newpage
\subsection{1.5}
有一铜块处于0℃和1 atm下,经测定,其膨胀系数和等温压缩系数分别为 $\alpha = 4.85 \times 10^{-5} K^{-1}, \kappa_T = 7.8 \times 10^{-7} (atm)^{-1}, \alpha$ 和 $\kappa_T$ 可近似当成常数。今使铜块加热至10℃,问:
(i) 压强要增加多少 atm才能维持铜块的体积不变?
(ii) 若压强增加100 atm,铜块的体积改变多少?

\newpage
\subsection{1.6}
已知一理想弹性丝的物态方程为
$$\mathcal{F} = bT \left( \frac{L}{L_0} - \frac{L_0^2}{L^2} \right),$$

其中 $\mathcal{F}$ 是张力;$L$ 是长度;$L_0$ 是张力为零时的 $L$ 值,$L_0$ 只是温度 $T$ 的函数;$b$ 是常数。定义 (线) 膨胀系数为
$$\alpha = \frac{1}{L} \left( \frac{\partial L}{\partial T} \right)_\mathcal{F},$$

等温杨氏模量为
$$ Y = \frac{L}{A} \left( \frac{\partial \mathcal{F}}{\partial L} \right)_T, $$

其中 $A$ 是弹性丝的截面积。证明:

(i) $$ Y = \frac{bT}{A} \left( \frac{L}{L_0} + \frac{2L_0^2}{L^2} \right); $$

(ii) $$ \alpha = \alpha_0 - \frac{1}{T} \frac{L^3}{L_0^3} \frac{L_0^3 - 1}{L_0^3 / L_0^3 + 2}, $$ 其中 $\alpha_0 = \frac{1}{L_0} \frac{dL_0}{dT}. $

\newpage
\subsection{1.7}
满足 $pV^n = C$ 的过程称为多方过程,其中 $n$ 和 $C$ 是常数,$n$ 称为多方指数。证明:

(i) 理想气体在多方过程中对外所作的功为
$$ (p_1 V_1 - p_2 V_2) / (n-1); $$

(ii) 理想气体在多方过程中的热容 $C_{(n)}$ 为
$$ C_{(n)} = \frac{n - \gamma}{n - 1} C_V, $$

其中 $\gamma = C_p / C_v$;

(iii) 当 $\gamma$ 为常数时,若一理想气体在某一过程中的热容量是常数,则这个过程一定是多方过程。

\newpage
\subsection{1.8}
抽成真空的小匣带有活门,打开活门让外面的空气冲入,当压强达到外界压强 $p_0$ 时将活门关上。

(i) 证明小匣内的空气在没有与外界交换热量之前,它的内能 $U$ 与原来在大气中的内能 $U_0$ 之差为 $U - U_0 = p_0 V_0$,其中 $V_0$ 是它原来在大气中的体积。

(ii) 若气体是理想气体且设 $\gamma = C_p / C_v$ 可近似当作常数,求它的温度 $T$ 与体积 $V$。

\newpage
\subsection{1.9}
一理想气体 $\gamma = C_p / C_v$ 是温度的函数,求在准静态绝热过程中 $T$ 与 $V$ 的关系和 $T$ 与 $p$ 的关系。这些关系中用到一个函数 $F(T)$,它由下式决定:
$$ \ln F(T) = \int \frac{dT}{(\gamma - 1)T}. $$

\newpage
\subsection{1.10}
利用上题的结果,证明当 $\gamma$ 是温度的函数时,理想气体卡诺循环的效率仍然是 $\eta = 1 - \frac{T_2}{T_1}$。

\newpage
\subsection{1.11}
10A的电流通过一个25Ω电阻器,历时1秒。
(i)若电阻器保持室温27℃不变,求电阻器的熵增加值;
(ii)电阻器被一绝热亮包起来,其初温为27℃,电阻器的质量为10g,定压比热为$c_p = 0.84J \cdot g^{-1} \cdot K^{-1}$,求电阻器的熵增加值。

\newpage
\subsection{1.12}
质量为 $m_1$、温度为 $T_1$ 的水与质量为 $m_2$、温度为 $T_2$ 的水在保持压强不变下混合,设水的定压比热 $c_p$ 可近似看成常数。证明熵增加为
$$c_p \left( (m_1 + m_2) \ln \frac{m_1 T_1 + m_2 T_2}{m_1 + m_2} - m_1 \ln T_1 - m_2 \ln T_2 \right)$$
当 $m_1 = m_2 = m$ 时简化为
$$mc_p \ln \frac{(T_1 + T_2)^2}{4 T_1 T_2}$$

\newpage
\subsection{1.13}
物体的初温 $T_1$ 高于热源的温度 $T_2$,令一热机工作于物体和热源之间,直到物体温度降低到 $T_2$ 为止。若热机从物体吸收的热量为 $Q$,试根据熵增加原理证明,此热机所能输出的最大功为
$$W_{\text{max}} = Q - T_2 (S_1 - S_2),$$
其中 $S_1 - S_2$ 是物体熵的减少值。

\newpage
\subsection{1.14}
有两个相同的物体,初始温度均为 $T_1$。令一致冷机工作于此两物体之间,使其中的一个物体温度降低到 $T_2$。设过程在定压下进行,且物体的 $C_p$ 可当作常数;降温过程中物体也没有相变发生。试根据熵增加原理证明,此过程所需的最小功为
$$W_{\text{min}} = C_p \left( \frac{T_1^2}{T_2} + T_2 - 2T_1 \right).$$







